% Inbuilt themes in beamer
\documentclass{beamer}

% Theme choice:
\usetheme{CambridgeUS}

% Title page details: 
\title{Assignment 6}
\subtitle{\Large AI1110: Probability and Random Variables \\ \large Indian Institute of Technology Hyderabad}
\author{Rudransh Mishra \\ \normalsize AI21BTECH11025 \\ \vspace*{20pt} \normalsize  15 June 2022 \\ \vspace*{20pt} PROBABILITY, RANDOM VARIABLES, AND STOCHASTIC PROCESSES\\ \normalsize Athanasios Papoulis}
\date{\today}
\logo{\large \LaTeX{}}


\begin{document}

% Title page frame
\begin{frame}
    \titlepage 
\end{frame}

% Remove logo from the next slides
\logo{}


% Question frame
\section{Question}
\begin{frame}{Question}
\begin{block}{Example 9-38}
Show that if R(r) is the inverse Fourier transform of a function $S(\omega)$ and $S(\omega) \geq 0$, then, for any a_i,\\
\begin{align}
&\sum_{i,k}^{} a_i a_k^{*} R(\tau_i - \tau_k) \geq 0
\end{align}
\end{block}
\end{frame}

% Solution frames
\begin{frame}{Proof}
\textbf{Proof.}
We know that,
\begin {align}
&\int_{-\infty}^{\infty}S(\omega) |\sum_{i}^{} a_i e^{j \omega \tau_i}|,d(\omega)\geq 0
\end {align}
And,
\begin {align}
&\int_{-\infty}^{\infty}S(\omega) | \sum_{i}^{} a_i e^{j \omega \tau_i}|,d(\omega) = \int_{-\infty}^{\infty}S(\omega) | \sum_{i,k}^{} a_i a_k^{*} e^{j \omega (\tau_i-\tau_k)}| ,d(\omega)
\end {align}
\end{frame}
\begin{frame}{Proof(contd.)}

\begin{center}
Now, we know that,

\begin {align}
&\int_{-\infty}^{\infty}S(\omega) | \sum_{i,k}^{} a_i a_k^{*} e^{j \omega (\tau_i-\tau_k)}| ,d(\omega)=\sum_{i,k}^{} a_i a_k^{*} R(\tau_i - \tau_k)
\end {align}
From (2) and (4),
\begin {align}
&\sum_{i,k}^{} a_i a_k^{*} R(\tau_i - \tau_k) \geq 0
\end {align}
and the proof is complete.
\end{center}

\end{frame}
\end{document}

\end{document}
